%! TEX program = lualatex
\documentclass[12pt,a4paper]{article} 

% Packages for formatting
\usepackage{fontspec}
\usepackage[ngerman]{babel}
\usepackage{geometry} 
\geometry{margin=1in} 
\usepackage{setspace} 
\usepackage{hyperref} 
\usepackage{xcolor}
\usepackage{amsmath, amssymb} % for align*
\usepackage{amsthm} % neue Theorem-Umgebungen
\usepackage{enumitem} % für schöne Listen (Teilaufgaben)
\usepackage{mathbbol}
\usepackage{graphicx}
\usepackage{amssymb}
\usepackage{gensymb}
\usepackage{stackengine}

\DeclareMathAlphabet\mathbb{U}{fplmbb}{m}{n}
% Style settings
\pagecolor{darkgray}      % sets background color to black 
\color{gray}          % sets text color to white

% Change subsection to use a, b, c instead of 1, 2, 3
\renewcommand{\thesubsection}{\alph{subsection})}
\newcommand{\zz}{\stackinset{c}{2.5pt}{c}{-2.5pt}{\textsf{Z}}{\textsf{Z}}}
% Title page info 
\title{Blatt 10}
\author{Hannes Rall \\ Albert-Ludwigs-University}
\date{\today}

\begin{document}
% Title page 
\begin{titlepage}
    \centering
    \vspace*{2cm}
    {\Huge\itshape Blatt 10\par}
    \vspace{2cm}
    {\Large\textsc{Hannes Rall}\par}
    \vfill
    {\large Albert-Ludwigs-University\\}
    \vspace{1cm}
    {\large\today\par}
\end{titlepage}

\newpage
\section*{Aufgabe 28}
\subsection*{(i) Füllen Sie den Lückentext:}
\zz : Für je zwei nichtparallele Geraden gibt es genau einen Punkt der auf diesen beiden Geraden liegt.\\
\
\begin{proof}
Seien $g$ und $h$ die zwei nichtparallelen Geraden. Nach Vorraussetzung (Geraden sind nichtparallel) muss es damit mindestens einen Punkt $p$ geben, der auf $g$ und $h$ liegt. Wir müssen nun zeigen, dass dies der einzige solche Punkt ist. Das zeigen wir mit einem Widerspruchsbeweis. Sei dazu auch q ein Punkt der auf g und h liegt mit $q \neq p$. Nach I1 gibt es genau eine Gerade, auf der $p$ und $q$ liegen, also muss $g=h$ sein, was den Widerspruch dazu gibt, dass g und h nichtparallel sind.
\end{proof}

\subsection*{(ii) Zeigen Sie die folgenden Aussagen und spezifizieren Sie dabei in jedem Schritt, welches Axiom sie
verwenden.}
\subsubsection*{(a)} 
\zz : Für jeden Punkt gibt es mindestens zwei verschiedene Geraden, so dass der Punkt auf beiden liegt.
\begin{proof}
Sei $P$ ein Punkt. Nach (I3) gibt es drei kollineare Punkte. Seien diese Punkte $A, B, C$. \\
\\
1. Fall: $P$ liegt auf einer der Geraden welche durch die kollinearen Punkte gehen. \\ 
Sei o.B.d.A. $P \in g_{AB}$. Dann gibt es nach (I1) eine Gerade $g_{PC}$ welche durch $P$ und $C$ geht und ungleich $g_{AB}$ sein muss, da sonst $A, B, C$ nicht kollinear wären. Somit ist $P \in g_{AB}$ und $P \in g_{PC}$. \\
\\
2. Fall: $P$ liegt nicht auf einer der Geraden welche durch die kollinearen Punkte gehen. \\
Dann sind $A, B, P$ (bzw. $A, C, P$ und $B, C, P$) kollinear.
Nach (I1) gibt es eine eindeutige Gerade $g_{AP}$ und eine eindeutige Gerade $g_{BP}$ welche ungleich sind, da $A, B, P$ kollinear und es ist $P \in g_{AP}$ und $P \in g_{BP}$.
\end{proof}

\newpage
\subsubsection*{(b)} 
\zz : Sei $g$ eine Gerade. Dann gibt es Geraden $h, l$, so dass $g, h$ und $l$ paarweise verschieden sind und sodass sowohl $h$ als auch $l$ die Gerade $g$ schneiden.
\begin{proof}
Sei $g$ eine Gerade. Seien $A, B, C$ wieder die drei kollinearen Punkte die es gibt nach (I3).\\
\\
1. Fall: Die Gerade $g$ geht durch o.B.d.A $A$ und $B$.\\
Nach (I1) gibt es genau eine Gerade durch zwei verschiedene Punkt, die Gerade $h = g_{AC}$ durch $A$ und $C$, sowie die Gerade $l = g_{BC}$ durch $B$ und $C$. Dadurch das $A, B, C$ kollinear sind, sind $g_{AB}, g_{AC}$ und $g_{BC}$ paarweise verschiedenen und es schneiden sich $g_{AB}$ und $g_{AC}$ im Punkt $A$ und $g_{AB}$ und $g_{BC}$ im Punkt $B$.\\
\\
2. Fall: Die Gerade $g$ verläuft durch keines der Punktpaare $A, B$; $A, C$ oder $B, C$.\\
Dann gibt es nach (I2) zwei Punkte $P, Q$ welche auf der Geraden liegen. Nun kann einer der Punkt $A, B, C$ noch auf einer Geraden liegen, jedoch nur genau einer. O.B.d.A sei $A$ nicht dieser Punkt. Dann gibt es nach (I1) eine Gerade $g_{PA}$ und eine Gerade $g_{QA}$. Diese Geraden schneiden $g$ in $P$ bzw $Q$ und sind verschieden, da $P \notin g_{QA}$ und $Q \notin g_{PA}$. Falls $B$ bzw. $C$ auf der Geraden g liegen, dann ist $P=B$ bzw $P=C$.
\end{proof}

\newpage
\section*{Aufgabe 29}
\subsubsection*{(i)}
\zz: Parallelität ist in affinen Ebenen eine Äquivalenzrelation auf der Menge der Geraden. \\
\begin{proof} Sei $g \sim_p h$, wenn $g \parallel h$. Seien $g$, $h$ und $k$ Geraden und es gilt $g \parallel h$ also $(g \sim_p h)$ und $h \parallel k$ also $(h \sim_p k)$.  \\
Nach Definition V.2.1. Geraden $g$ und $h$ sind parallel, wenn $g$ und $h$ keinen gemeinsamen Punkt haben oder $g = h$. \\
\\
Reflexivität: \\
Da $g=g$, folgt $g \sim_p g$.\\
\\
Symmetrie: \\
Es gilt $g \sim_p h$. Wenn $g$ und $h$ keine gemeinsamen Punkte haben, so haben auch $h$ und $g$ keine gemeinsamen Punkte und ist $h \parallel g$ also $h \sim_p g$. Wenn $g=h$, dann ist $h=g$ und es folgt $h \parallel g$, also $h \sim_p g$.\\
\\
Transitivität:\\
Es gilt $g \sim_p h$ und $h \sim_p k$. Angenommen $g$ und $k$ wären nichtparallel. Dann hätten sie nach Aufgabe 28 einen gemeinsamen Punkt. Sei $P$ dieser Punkt. Nach Parallelenaxiom gibt es genau eine parallele Gerade zu $h$ die den Punkt $P$ enthält und dann wäre $g=k$ was ein Wiederspruch wäre zu $g$ und $k$ sind nichtparallel.
\end{proof}

\subsubsection*{(ii)}
\zz: Eine affine Ebene ist eine Inzidenzgeometrie.
\begin{proof}
Per Definition ist eine affine Ebene eine Inzidenzstruktur welche (I1), (I3) und das strenge Parallelenaxiom erfüllt. Nun muss noch gezeigt werden, dass dann auch (I2) gilt. \\
Angenommen, es gibt eine Gerade $g \in G$ mit $g = \{P\}$, also enthält $g$ nur den Punkt $P$.\\
Nach (I3) existieren mindestens drei Punkte $A, B, C \in P$, die nicht auf einer gemeinsamen Geraden liegen.\\

\end{proof}
\end{document}
